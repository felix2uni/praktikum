\section{Auswertung}
\label{sec:Auswertung}

\section{Auswertung}
\subsection{Fouriersynthese}
\begin{figure}
\begin{subfigure}{0.5\textwidth}
    \centering
    \includegraphics[width=\textwidth]{assets\fourier_zahn.pdf}
    \caption{Rechteckschwingung}
\end{subfigure}
\begin{subfigure}{0.5\textwidth}
    \centering
    \includegraphics[width=\textwidth]{assets\fourier_zahn.pdf}
    \caption{Dreieckschwingung}
\end{subfigure}
\begin{subfigure}{0.5\textwidth}
    \centering
    \includegraphics[width=\textwidth]{assets\fourier_zahn.pdf}
    \caption{Sägezahnschwingung}
\end{subfigure}
\caption{Mit dem Oszilloskop aufgezeichnete Schwingungsbilder}
\label{fig:oszi}
\end{figure}

In Abbildung \ref{fig:oszi} sind die aufgezeichneten Schwingungsbilder für die drei eingestellten Schwingungen dargestellt. Es zeigt sich, dass alle drei in guter Übereinstimmung mit den in der Vorbereitung vorhergesagten Kurven stehen.

\subsection{Fourieranalyse}
\subsubsection{Rechteckschwingung}
\begin{table}[h]
	\centering
	\caption{Aufgenommene Messwerte zur Rechteckspannung}
	\label{tab:rechteck_messwerte}
	\begin{tabular}{ S S S S }
		\toprule
		{ $\text{Oberschwingung} \: n $ } & { $ \nu \: / \: \si{kHz} $} & {$ \text{Amplitude} \: / \: \si{dB} $} & {$ \text{Amplitude}\: U \: / \: \si{V} $}\\
		\midrule
            1 & 100 & 39.0 & 89.12 \\ 
            3 & 300 & 29.4 & 29.51 \\
            5 & 500 & 25.0 & 17.78 \\
            7 & 700 & 22.2 & 12.88 \\
            9 & 900 & 20.6 & 10.71 \\
            11 & 1100 & 19.0 & 8.91 \\
            13 & 1300 & 17.8 & 7.76 \\
            15 & 1500 & 16.6 & 6.76 \\
            17 & 1700 & 16.2 & 6.45 \\
            19 & 1900 & 15.4 & 5.88 \\
	\end{tabular}
\end{table}
In Tabelle \ref{tab:rechteck_messwerte} sind die vom Oszilloskop abgelesenen Positionen und Höhen der Peaks der Rechteckschwingung aufgetragen. Wie in der Vorbereitung vorhergesagt, lassen sich keine peaks für gerade $n$ messen.
Zur Überprüfung der $n$-Abhängigkeit ist in Abbildung \ref{fig:rechteck_fit} die Höhe der Peaks einmal direkt gegen die Frequenz und einmal gegen $\frac{1}{n}$ aufgetragen. Bei letzterem ist zu sehen, dass alle Messwerte näherungsweise auf einer Geraden liegen. Der Fit ist eine Gerade der Form
\begin{equation*}
    y = ax + b
\end{equation*}
mit den Parametern
\begin{align*}
    a = \SI{88.011 \pm 0.520}{V}
    b = \SI{0.805 \pm 0.181}{V} .
\end{align*}
\begin{figure}[h]
\begin{subfigure}{0.5\textwidth}
    \centering
    \includegraphics[width=\textwidth]{assets\rechteck_messung.pdf}
\end{subfigure}
\begin{subfigure}{0.5\textwidth}
    \centering
    \includegraphics[width=\textwidth]{assets\rechteck_fit.pdf}
\end{subfigure}
\caption{Messdaten für die Rechteckschwingung}
\label{fig:rechteck_fit}
\end{figure}

\subsubsection{Dreieckschwingung}
\begin{table}[h]
	\centering
	\caption{Aufgenommene Messwerte zur Dreieckspannung}
	\label{tab:dreieck_messwerte}
	\begin{tabular}{ S S S S }
		\toprule
		{ $\text{Oberschwingung} \: n $ } & { $ \nu \: / \: \si{kHz} $} & {$ \text{Amplitude} \: / \: \si{dB} $} & {$ \text{Amplitude}\: U \: / \: \si{V} $}\\
		\midrule
            1 & 100 & 35.00 & 56.23 \\ 
            3 & 300 & 16.20 & 6.46 \\
            5 & 500 & 7.01 & 2.241 \\
            7 & 700 & 1.81 & 1.23 \\
            9 & 900 & -2.99 & 0.71 \\
            11 & 1100 & -6.59 & 0.47 \\
            13 & 1300 & -8.59 & 0.37 \\
	\end{tabular}
\end{table}
In Tabelle \ref{tab:dreieck_messwerte} sind wieder die abgelesenen Positionen und Höhen der Peaks der Dreieckschwingung aufgetragen. Wie bei der Rechteckschwingung, lassen sich keine Peaks für gerade $n$ messen. Wie zuvor sind in Abb. \ref{fig:dreieck_fit} die Höhen der Peaks einmal gegen die Frequenzen und diesmal gegen $\frac{1}{n^2}$ aufgetragen.
Letztere Abbildung zeigt wieder, dass die Messwerte näherungsweise auf einer Geraden liegen. Die Ausgleichsgerade hat die Parameter
\begin{align*}
    a = \SI{56.193 \pm 0.092}{V}
    b = \SI{0.056 \pm 0.035}{V} .
\end{align*}
\begin{figure}[h]
\begin{subfigure}{0.5\textwidth}
    \centering
    \includegraphics[width=\textwidth]{assets\dreieck_messung.pdf}
\end{subfigure}
\begin{subfigure}{0.5\textwidth}
    \centering
    \includegraphics[width=\textwidth]{assets\dreieck_fit.pdf}
\end{subfigure}
\caption{Messdaten für die Dreieckschwingung}
\label{fig:dreieck_fit}
\end{figure}
\subsubsection{Sägezahnschwingung}
\begin{table}[h]
	\centering
	\caption{Aufgenommene Messwerte zur Sägezahnspannung}
	\label{tab:zahn_messwerte}
	\begin{tabular}{ S S S S }
		\toprule
		{ $\text{Oberschwingung}\: n $ } & { $ \nu \: / \: \si{kHz} $} & {$ \text{Amplitude} \: / \: \si{dB} $} & {$ \text{Amplitude}\: U \: / \: \si{V} $}\\
		\midrule
            1 & 100 & 32.8 & 43.65 \\ 
            2 & 205 & 27.0 & 22.38 \\
            3 & 305 & 22.6 & 13.49 \\
            4 & 405 & 21.4 & 11.75 \\
            5 & 505 & 19.8 & 9.77 \\
            6 & 605 & 18.2 & 8.13 \\
            7 & 705 & 17.4 & 7.41 \\
            8 & 810 & 16.6 & 6.76 \\
            9 & 910 & 15.8 & 6.16 \\
            10 & 1010 & 15.4 & 5.89 \\
	\end{tabular}
\end{table}
Tabelle \ref{tab:zahn_messwerte} zeigt wieder die Positionen und Höhen der Peaks. Anders als bei den vorherigen Schwingungen sind hier auch Peaks für gerade $n$ messbar. In Abb. \ref{fig:zahn_fit} sind die Werte gegen die Frequenzen und gegen $\frac{1}{n}$ aufgetragen. Es lässt sich, wie vorhergesagt, ein linearer Zusammenhang zwischen der Höhe der Peaks und $\frac{1}{n}$ feststellen. Die Ausgleichsgerade hat die Parameter
\begin{align*}
    a = \SI{42.068 \pm 0.812}{V}
    b = \SI{1.219 \pm 0.320}{V} .
\end{align*}
\begin{figure}[h]
\begin{subfigure}{0.5\textwidth}
    \centering
    \includegraphics[width=\textwidth]{assets\zahn_messung.pdf}
\end{subfigure}
\begin{subfigure}{0.5\textwidth}
    \centering
    \includegraphics[width=\textwidth]{assets\zahn_fit.pdf}
\end{subfigure}
\caption{Messdaten für die Sägezahnschwingung}
\label{fig:zahn_fit}
\end{figure}
  \caption{Fourier-Annäherung einer Sägezahnschwingung für $n=9$}
  \label{fig:vorbereitung_sägezahn}
\end{figure}
