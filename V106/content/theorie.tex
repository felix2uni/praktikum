\section{Theorie}
Im folgenden Versuch werden verschiedene Schwingungsarten von gekoppelten Pendeln betrachtet. Das Pendel soll in diesem Versuch als ein mathematisches Pendel der Länge $l$ und der Masse $m$ angenommen und betrachtet werden.
Die Bewegungsgleichung eines mathematisches Pendel lässt sich aus dem zweiten Newtonschen Axiom herleiten,
\begin{equation}
  F=m\cdot\ddot{x} \label{eqn:newton2}
\end{equation}
indem man die Gleichung \eqref{eqn:newton2} und die rücktreibende Kraft des Systems gleichsetzt.
\begin{equation}
  F_\text{rück}=-m\cdot g\cdot sin(\phi(t)) \label{eqn:pendelkraft}
\end{equation}