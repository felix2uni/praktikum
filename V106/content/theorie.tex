\section{Theorie}
Im folgenden Versuch werden verschiedene Schwingungsarten von gekoppelten Pendeln betrachtet. Das Pendel soll in diesem Versuch als ein mathematisches Pendel der Länge $l$ und der Masse $m$ angenommen und betrachtet werden.
Die Bewegungsgleichung eines mathematisches Pendel lässt sich aus dem zweiten Newtonschen Axiom herleiten,
\begin{equation}
  F=m\cdot\ddot{x} \label{eq:newton2}
\end{equation}
indem man die Gleichung \eqref{eq:newton2} und die rücktreibende Kraft des Systems gleichsetzt.
\begin{equation}
  F_\text{rück}=-m\cdot g\cdot \sin(\phi(t)) \label{eq:pendelkraft}
\end{equation}
In \eqref{eq:newton2} muss jetzt noch die Beschleunigung $\ddot x$ an das Problem angepasst werden. Diese ist bei einem Pendel die tangentiale
Beschleunigung. Es gilt:
\begin{equation}
  \ddot x(t)=l\cdot \ddot{\phi}(t)
\end{equation}
In \eqref{eq:newton2} muss jetzt noch die Beschleunigung $\ddot x$ an das Problem angepasst werden. Diese ist bei einem Pendel die tangentiale
Beschleunigung. Es gilt:
\begin{equation}
  \ddot x(t)=l\cdot \ddot{\phi}(t)
\end{equation}
Der Winkel $\phi(t)$ beschreibt dabei die Auslenkung des Pendels zum Zeitpunkt $t$. Gleichsetzen von \eqref{eq:newton2} und
\eqref{eq:pendelkraft} ergibt nun die Differentialgleichung
\begin{eqnarray}
  \ddot{\phi}(t)+\frac{g}{l}\sin(\phi(t))=0.
\end{eqnarray}
Die Differentialgleichung ist auf Grund des Sinus jedoch nicht linear und deswegen weder analytisch lösbar, noch ein harmonischer Oszillator.
Um die Differentialgleichung trotzdem lösen zu können, wird hier die Taylorfunktion des Sinus betrachtet
\begin{equation}
  \sin(x)=\sum_{n=0}^\infty (-1)^n\frac{x^{2n+1}}{(2n+1)!}.
\end{equation}
Für $n=0$ ergibt sich die sogenannte Kleinwinkelnäherung
\begin{equation}
  \sin(x)\approx x. \label{eq:kleinwinkel}
\end{equation}
Durch diese wird zumindest für kleine Winkel $\phi$ die lineare Differentialgleichung des harmonischen Oszillators
\begin{equation}
  \ddot{x}+\omega^2\cdot x=0 \label{eq:harmOsz}
\end{equation}
erlangt. Zwar entsteht bei der Kleinwinkelnäherung ein Fehler, dieser ist für kleine Winkel jedoch so gering, dass im Folgenden die Näherung für die weiteren Berechnungen verwendet wird.
Die folgenden Rechnungen sind also nur unter der Annahme einer kleinen Auslenkung sinnvoll.
Die Differentialgleichung lautet nun
\begin{equation}
  \ddot{\phi}(t)+\frac{g}{l}\phi(t)=0
\end{equation}
und wird durch den Ansatz
\begin{align}
  x(t)        & =A\cdot \cos(\omega\cdot t)+B\cdot \sin(\omega\cdot t) \\
  \ddot{x}(t) & =-\omega^2\cdot x(t)
  \label{eq:ansatz}
\end{align}
gelöst. Die Frequenz ergibt sich dabei zu
\begin{equation}
  \omega=\sqrt{\frac{g}{l}}
\end{equation}
und mittels
\begin{equation}
  T=\frac{2\times\pi}{\omega}
\end{equation}
ergibt sich die Periodendauer
\begin{equation}
  T=2 \times \pi \times \sqrt{\frac{l}{g}}.
  \label{eq:Periodendauer}
\end{equation}
Die Frequenz des Pendels ist also sowohl unabhängig von der Masse $m$, als auch von der Auslenkung, solange diese hinreichend klein ist.

\subsection{Gekoppelte Pendel}
Werden zwei identische Pendel mit einer Feder gekoppelt, so schwingen diese nicht mehr unabhängig von einander, da die Pendel zusätzliche Kräfte auf
einander auswirken. Als zusätzlichen Größe fließt die Federkonstante $k$ in die Gleichung ein. Zudem sind nun zwei Längen für das System relevant, nämlich die
Länge der Pendel $L$ und der Abstand $l$, mit dem die Feder an den Pendeln befestigt ist.
\\
Es liegt nun ein System aus zwei Differentialgleichungen vor, die zudem miteinander gekoppelt sind:
\begin{align}
  J\ddot{\phi}_1 & =-mgL\phi_1+kl^2(\phi_2-\phi_1) \\
  J\ddot{\phi}_2 & =-mgL\phi_2+kl^2(\phi_2-\phi_1)
\end{align}
Das Differentialgleichungs-System wird im Folgenden nur für drei verschiedene Anfangsbedingungen gelöst, da nur diese für den Versuch relevant sind.
Die Anfangsbedingungen sind hierbei die Winkel $\alpha_1$ und $\alpha_2$, mit denen die Pendel zu Beginn ausgelenkt werden, und die Geschwindigkeiten
$v_1$ und $v_2$, die die Pendel zu Beginn erhalten. Für die drei relevanten Fälle gilt $v_1=v_2=0$.

\begin{enumerate}
  \item \textit{Gleichsinnige Schwingung: $\alpha_1=\alpha_2$} \\
        Bei einer gleichsinnigen Schwingung werden beide Pendel um den gleichen Winkel ausgelenkt. Die Pendel schwingen nun unabhängig von einander,
        so wie sie auch ohne eine Kopplung schwingen würden. Deswegen ist auch die Frequenz für mathematische Pendel analog zu jener, die wir für einzele
        mathematische Pendel her geleitet haben
        \begin{equation}
          \omega_{+}=\sqrt{\frac{g}{l}}.
        \end{equation}
  \item \textit{Gegensinnige Schwingung: $\alpha_1=-\alpha_2$}\\
        Bei einer gegensinnigen Schwingung werden beide Pendel genau entgegengesetzt ausgelenkt. Der Betrag des Winkels ist also exakt gleich groß, das Vorzeichen
        ist jedoch genau entgegen gesetzt. Die Feder übt nun eine Kraft auf die Pendel aus, die abhängig von der Auslenkung der Pendel ist. Dadurch ergibt sich die
        Frequenz für mathematische Pendel zu
        \begin{equation}
          \omega_{-}=\sqrt{\frac{g}{l}+2\frac{kl^2}{mL^2}}.
        \end{equation}
        Die Frequenz hängt nun also auch noch von der Federkonstante $k$ ab.
  \item \textit{Gekoppelte Schwingung: $\alpha_1=0$, $\alpha_2\neq 0$}\\
        Bei einer gekoppelten Schwingung wird nur eines der beiden Pendel ausgelenkt. Das andere Pendel bleibt in der Ruhelage. Nun kommt es zu dem Phänomen der
        Schwebung. Das erste Pendel, welches ausgelenkt wird, überträgt bei seiner Schwingung seine Energie nach und nach an das zweite Pendel, welches sich
        zu Beginn in der Ruhelage befand. Dadurch fängt nun das zweite Pendel an zu schwingen. Hat das erste Pendel seine gesamte Energie an das zweite Pendel
        abgegeben, kommt es in der Ruhelage zum Stillstand. Das zweite Pendel, welches nun so schwingt wie das Erste zu Beginn, überträgt nun seine Energie wieder
        an das erste Pendel, bis sich der Vorgang abermals wiederholt. Die sogenannte Schwebungsfrequnz beschreibt nun die Zeitspanne zwischen dem Stillstand eines
        Pendels bis zum nächsten Stillstand des selben Pendels. Die Schwebungsfrequnz ist dabei gegeben durch
        \begin{equation}
          \omega_\text{S}=\omega_+ - \omega_-
        \end{equation}
        und die Schwebungsdauer ist gegeben durch
        \begin{equation}
          T_\text{S}=\frac{T_+\cdot T_-}{T_+-T_-}.
        \end{equation}
        \\
        Um die Kopplung zwischen den beiden Pendeln zu beschreiben, wird die Kopplungskonstante $K$ definiert
        \begin{equation}
          \label{eq:kopplungskonstante}
          K=\frac{\omega_-^2-\omega_+^2}{\omega_-^2+\omega_+^2}=\frac{T_-^2-T_+^2}{T_-^2+T_+^2}.
        \end{equation}
\end{enumerate}