\section{Diskussion}
In Tabelle \ref{tab:diskussion} sind alle experimentell bestimmten Schwingungsdauern, sowie die relativen Abweichungen $\Delta T$ von den Theoriewerten aufgelistet.
\begin{table}
	\centering
	\caption{Vergleich der experimentell ermittelten Schwingungsdauern mit den Theoriewerten.}
	\label{tab:diskussion}
	\begin{tabular}{ S | S S | S S  }
		\toprule
		{ } & \multicolumn{2}{c}{$L_1$} & \multicolumn{2}{c}{$L_2$} \\
		\cmidrule(lr){2-3}\cmidrule(lr){4-5}
		{ } & {$ T \:/\: \si{s}$} & {$ \symup{\Delta} T $} & {$ T\:/\: \si{s}$} & {$ \symup{\Delta} T $}  \\
		\midrule
		$T_\text{th}$ & 1.419 \pm 0.014 &  & 1.269 \pm 0.016 &  \\
		$T_\text{a}$  & 1.445 \pm 0.003 & 2.70 \% & 1.356 \pm 0.005 & 6.78 \% \\
		$T_\text{b}$  & 1.452 \pm 0.004 & 2.36 \% & 1.353 \pm 0.005 & 6.62 \% \\
		\midrule
		$T_\text{+, th}$ & 1.419 \pm 0.014 &  & 1.269 \pm 0.016 &  \\
		$T_\text{+, exp}$  & 1.445 \pm 0.003 & 3.31 \% & 1.356 \pm 0.005 & 6.93 \% \\
		\midrule
		$T_\text{-, th}$ & 1.419 \pm 0.014 &  & 1.269 \pm 0.016 &  \\
		$T_\text{-, exp}$  & 1.445 \pm 0.003 & 5.20 \% & 1.356 \pm 0.005 & 1.66 \% \\
		\midrule
		$T_\text{s, th}$ & 1.419 \pm 0.014 &  & 1.269 \pm 0.016 &  \\
		$T_\text{s, exp}$  & 1.445 \pm 0.003 & 8.80 \% & 1.356 \pm 0.005 & 8.22 \% \\
		$T_\text{s, ber}$  & 1.445 \pm 0.003 & 2.94 \% & 1.356 \pm 0.005 & 3.22 \% \\
		\bottomrule
	\end{tabular}
\end{table}
Die experimentell bestimmten Schwingungsdauern der Pendel weichen von den theoretisch vorhergesagten Werten ab.
Mögliche Gründe für die Abweichung liegen zum einen in Vereinfachungen, die zum Bestimmen des Theoriewertes vorgenommen wurden. Die theoretische Vorhersage gilt für ein perfektes 
mathematisches Pendel, also ein Pendel an einer masselosen Pendelstange, das ohne Reibung in einer Ebene schwingt. Im Experiment konnten weder die masselose Stange noch die 
Reibungsfreiheit realisiert werden. Eine andere mögliche Fehlerquelle ist die Art der Zeitenbestimmung. Im Versuch wurde die Stoppuhr gestartet und gestoppt, wenn das Pendel 
das Auslenkungsmaximum erreicht hat. Dieser Punkt konnte mit dem bloßen Auge allerdings nicht immer präzise bestimmt werden, wodurch Abweichungen im Messvorgang entstehen. Trotz 
dieser Umstände ist zu sehen, dass bei beiden Pendellänge die Kopplungskonstante der Feder zu einem nahezu identische wert bestimmt wurde. Auch bei der Betrachtung der 
Schwebungsdauern ist zu sehen, dass sowohl die gemessenen als auch die berechneten Werte im Rahmen der Messunsicherheiten übereinstimmen.