\section{Auswertung}
\label{sec:Auswertung}

Der erste Teil des Versuches wurde mit den Pendellängen
\begin{equation*}
    L_1 = \SI{50 +- 1}{cm} 
\end{equation*}
durchgeführt. Der Fehler wurde dabei mit 1 cm abgeschätzt. Da die gemessenen Werte für die Schwingungsdauer bei beiden Pendeln im Rahmen der Messungenauigkeit übereinstimmten,
wurden diese Pendellänge für die folgenden Teile beibehalten.
Der zweite Teil des Versuches wurde mit der Pendellänge
\begin{equation*}
    L_2 = \SI{40 +- 1}{cm}
\end{equation*}
durchgeführt.

\subsection{Justierung}


\begin{table}
	\centering
	\caption{Experimentell ermittelte Schwingungsdauern}
	\label{tab:justierung}
	\begin{tabular}{ S S | S S  }
		\toprule
		\multicolumn{2}{c}{$L_1$} & \multicolumn{2}{c}{$L_2$} \\
		\cmidrule(lr){1-2}\cmidrule(lr){3-4}
		{$ T_\text{a}\:/\: \si{s}$} & {$ T_\text{b}\:/\: \si{s}$} & {$ T_\text{a}\:/\: \si{s}$} & {$ T_\text{b}\:/\: \si{s}$}  \\
		\midrule
1.454 & 1.452 & 1.340 & 1.316 \\
1.448 & 1.448 & 1.340 & 1.366 \\
1.458 & 1.482 & 1.384 & 1.338 \\
1.460 & 1.436 & 1.368 & 1.350 \\
1.458 & 1.448 & 1.368 & 1.360 \\
1.442 & 1.460 & 1.368 & 1.346 \\
1.442 & 1.448 & 1.350 & 1.372 \\
1.448 & 1.452 & 1.350 & 1.356 \\
1.436 & 1.436 & 1.344 & 1.374 \\
1.452 & 1.454 & 1.344 & 1.350 \\

		\bottomrule
	\end{tabular}
\end{table}
Da bei der Durchführung die Dauer von 5 Schwingungen gemessen wurde, müssen die Messwerte durch 5 geteilt werden, um die Schwingungsdauer T zu erhalten.
Die so berechneten Werte für $T$ sind in Tabelle \ref{tab:justierung} dargestellt.
Durch die Mittelwertsformel 
\begin{equation}
    \label{eq:mittelwert}
    \bar{x} = \frac{1}{N} \sum_{n=1}^N x_n
\end{equation}
und die Formel für die Standardabweichung
\begin{equation}
    \label{eq:abweichung}
    \Delta x = \sqrt{\frac{1}{N(N-1)} \sum_{n=1}^N (x_n-\bar{x})^2 }
\end{equation}
ergeben sich für $L_1$ die Mittelwerte
\begin{gather*}
    T_\text{a} = \SI{1.4498 +- 0.0025}{s} \\
    T_\text{b} = \SI{1.4520 +- 0.0040}{s}
\end{gather*}
$T_\text{a}$ ist dabei die Periodendauer des ersten Pendels, $T_\text{b}$ die Periodendauer des zweiten.

\noindent Durch Gl.\eqref{eq:Periodendauer} kann für $L = \SI{50 +- 1}{cm} $ und $g = \SI{9.81}{\frac{m}{s^2}} $ der Theoriewert für die Schwingungsdauer zu
\begin{equation*}
    T_\text{th} = \SI{1.419 +- 0.014}{s}
\end{equation*}
bestimmt werden.
Der Fehler berechnet sich dabei nach
\begin{equation}
    \label{eq:theorie_fehler}
    \symup{\Delta} T = \sqrt{ \left( \frac{1}{2} \sqrt{ \frac{1}{g l} } \right)^2 ( \symup{\Delta L)}  ^2 } .
\end{equation}
Analog ergeben sich für $L_2$ die Werte:
\begin{gather*}
    T_\text{a} = \SI{1.355 +- 0.004}{s} \\
    T_\text{b} = \SI{1.353 +- 0.005}{s}
\end{gather*}
Der dazugehörige Theoriewert ist 
\begin{equation*}
    T_\text{th} = \SI{1.269 +- 0.016}{s}
\end{equation*}
Der Fehler wurde mit Gl.\eqref{eq:theorie_fehler} brerechnet.


\subsection{Bestimmung der Kopplungskonstante}
\label{sec:kopplung}
\begin{table}
	\centering
	\caption{Experimentell ermittelte Schwingungsdauern $T_+$ und $T_-$}
	\label{tab:TPlusMinus}
	\begin{tabular}{ S S | S S  }
		\toprule
		\multicolumn{2}{c}{$L_1$} & \multicolumn{2}{c}{$L_2$} \\
		\cmidrule(lr){1-2}\cmidrule(lr){3-4}
		{$ T_+\:/\: \si{s}$} & {$ T_-\:/\: \si{s}$} & {$ T_+\:/\: \si{s}$} & {$ T_-\:/\: \si{s}$}  \\
		\midrule
        1.466 & 1.374 & 1.338 & 1.280 \\
        1.448 & 1.406 & 1.356 & 1.292 \\
        1.482 & 1.418 & 1.386 & 1.286 \\
        1.482 & 1.402 & 1.350 & 1.320 \\
        1.470 & 1.408 & 1.340 & 1.316 \\
        1.460 & 1.436 & 1.360 & 1.276 \\
        1.454 & 1.414 & 1.350 & 1.340 \\
        1.470 & 1.378 & 1.358 & 1.294 \\
        1.458 & 1.418 & 1.368 & 1.322 \\
        1.470 & 1.386 & 1.366 & 1.288 \\

		\bottomrule
	\end{tabular}
\end{table}
\paragraph{1. Für L1:}
In Tabelle \ref{tab:TPlusMinus} sind die experimentell gemessenen Werte für $T_+$ und $T_-$ für beide Pendellängen aufgelistet.
Wie zuvor wurden dabei die gemessenen Zeiten durch $5$ geteilt, um die Periodendauer zu bestimmen. Durch Mittelwertsbildung und Bestimmung der
Standardabweichung ergeben sich:
\begin{gather*}
    T_+ = \SI{1.4660 +- 0.0035}{s} \\
    T_- = \SI{1.404 +- 0.006}{s}
\end{gather*}
Durch Umstellen von Gl. \eqref{eq:schw} zu
\begin{equation}
    \label{eq:frequenz}
    \omega = \frac{2 \pi}{T}
\end{equation}
können aus den ermittelten Schwingungsdauern die zugehörigen Frequenzen zu
\begin{align*}
    \omega_+ = \SI{4.286 +- 0.010}{\frac{1}{s}} \\
    \omega_- = \SI{4.475 +- 0.020}{\frac{1}{s}} 
\end{align*}
berechnet werden.
Mit Gleichung \eqref{eq:kopplungskonstante} lässt sich daraus die Kopplungskonstante $K$ bestimmen zu 
\begin{equation*}
    K = \num{0.043 +- 0.005} .
\end{equation*}
Mit Hilfe von Gl. \eqref{eq:omega_plus} lässt sich aus der Pendellänge und $g$ ein Theoriewert für $\omega_+$ berechnen, aus Gl. \eqref{eq:omega_minus} lässt sich mit Hilfe der zuvor
bestimmten Kopplungskonstante ein Theoriewert für $\omega_-$ bestimmen. Diese Werte werden nach Gl. \eqref{eq:schw} zusätzlich in Schwingungsdauern umgerechnet, sodass sich die Theoriewerte
\begin{align*}
    \omega_\text{+, th} = \SI{4.43 +- 0.04}{\frac{1}{s}} , \\
    T_\text{+, th} = \SI{1.419 +- 0.014}{s} \\
    \omega_\text{-, th} = \SI{4.24 +- 0.05}{\frac{1}{s}} , \\
    T_\text{-, th} = \SI{1.481 +- 0.017}{s} 
\end{align*}
ergeben.
\paragraph{2. Für L2:}
Die gemessenen Schwingungsdauern sind in Tabelle \ref{tab:TPlusMinus} zu sehen. Durch Mittelung der Werte ergeben sich die Schwingungsdauern
\begin{gather*}
    T_+ = \SI{1.357 +- 0.004}{s} \\
    T_- = \SI{1.301 +- 0.007}{s} .
\end{gather*}
Durch Nutzung von Gl. \eqref{eq:frequenz} ergeben sich daraus die Frequenzen
\begin{align*}
    \omega_+ = \SI{4.630 +- 0.015}{\frac{1}{s}} \\
    \omega_- = \SI{4.828 +- 0.025}{\frac{1}{s}} .
\end{align*}
Mit Gleichung \eqref{eq:kopplungskonstante} lässt sich die Kopplungskonstante $K$ bestimmen zu
\begin{equation*}
    K = \num{0.042 +- 0.006} .
\end{equation*}
Nach Gleichungen \eqref{eq:omega_plus}, \eqref{eq:omega_minus} und \eqref{eq:schw} ergeben sich die Theoriewerte
\begin{align*}
    \omega_\text{+, th} = \SI{4.95 +- 0.06}{\frac{1}{s}} , \\
    T_\text{+, th} = \SI{1.269 +- 0.016}{s} \\
    \omega_\text{-, th} = \SI{4.75 +- 0.07}{\frac{1}{s}} , \\
    T_\text{-, th} = \SI{1.323 +- 0.018}{s} 
\end{align*}
\subsection{Messung der Schwebungsdauer}

\begin{table}
	\centering
	\caption{Experimentell ermittelte Schwebungsdauern $T_\text{s}$ und Schwingungsdauern $T$}
	\label{tab:schwebung}
	\begin{tabular}{ S S | S S  }
		\toprule
		\multicolumn{2}{c}{$L_1$} & \multicolumn{2}{c}{$L_2$} \\
		\cmidrule(lr){1-2}\cmidrule(lr){3-4}
		{$ T_\text{s}\:/\: \si{s}$} & {$ T \:/\: \si{s}$} & {$ T_\text{s}\:/\: \si{s}$} & {$ T \:/\: \si{s}$}  \\
		\midrule
		37.402 & 1.470 & 28.160 & 1.332  \\ 
		37.404 & 1.492 & 28.632 & 1.346  \\ 
		37.526 & 1.488 & 28.368 & 1.316  \\ 
		37.578 & 1.566 & 28.506 & 1.306  \\ 
		37.526 & 1.452 & 28.762 & 1.310  \\ 
		37.512 & 1.436 & 28.510 & 1.332  \\ 
		37.464 & 1.482 & 28.064 & 1.316  \\ 
		37.494 & 1.466 & 28.460 & 1.316  \\ 
		37.734 & 1.446 & 28.636 & 1.310  \\ 
		37.424 & 1.460 & 28.386 & 1.320  \\ 
		
		\bottomrule
	\end{tabular}
\end{table}

\paragraph{1. Für L1: }
In Tabelle \ref{tab:schwebung} sind die gemessenen Schwebungsdauern und Schwingungsdauern für beide Pendellängen aufgelistet. Die Werte wurden wie zuvor durch 5 geteilt, um die Dauer einer Schwebung zu bestimmen.
Der Mittelwert der Schwingungsdauer und die daraus resultierende Schwingungsfrequenz für die Pendellänge $L_1$ ergeben sich zu
\begin{align*}
    T = \SI{1.476 +- 0.012}{s}, \\
    \omega = \SI{4.257 +- 0.033}{\frac{1}{s}} .
\end{align*}
Für die Pendellänge $L_1$ ergibt sich der Mittelwert
\begin{equation*}
    T_\text{s} = \SI{37.506 +- 0.031}{s}
\end{equation*}
für die Schwebungsdauer, nach Gl. \eqref{eq:schw} ergibt sich daraus für die Schwebungsfrequenz
\begin{equation*}
    \omega_\text{s} = \SI{0.167 +- 0.001}{\frac{1}{s}} .
\end{equation*}
Die Schwebungsdauer lässt sich nach Gl.\eqref{eq:schwebungsdauer} auch mit Hilfe der in \ref{sec:kopplung} ermittelten Schwingungsdauern $T_+$ und $T_-$ berechnen zu
\begin{equation*}
    T_\text{s} = \SI{33.0 +- 4.0}{s}, 
\end{equation*}
die dazugehörige Frequenz ist
\begin{equation*}
    \omega_\text{s} = \SI{0.189 +- 0.022}{\frac{1}{s}} .
\end{equation*}
Der Fehler wurde dabei gemäß der gausschen Fehlerfortpflanzung berechnet nach
\begin{equation*}
    \symup{\Delta} T_s = \sqrt{ \left( \frac{- T_-^2}{(T_+-T_-)^2} \right)^2 (\symup{\Delta} T_+)^2 + \left( \frac{T_+^2}{ (T_+ - T_- )^2}  \right)^2 (\symup{\Delta} T_-)^2 } .
\end{equation*}
Mit den zuvor bestimmten Theoriewerten lässt sich nach Gl. \eqref{eq:schwebungsdauer} auch ein theoretischer Wert für die Schwebungsdauer, und daraus ein Wert für die Schwebungsfrequenz
berechnen:
\begin{align*}
    T_\text{s, th} = \SI{34 +- 4}{s} \\
    \omega_\text{s, th} = \SI{0.187 +- 0.021}{\frac{1}{s}} .
\end{align*}

\paragraph{2. Für L2: }
Der Mittelwert der Schwingungsdauer und die daraus resultierende Schwingungsfrequenz ergeben sich zu
\begin{align*}
    T = \SI{1.320 +- 0.004}{s}, \\
    \omega = \SI{4.759 +- 0.014}{\frac{1}{s}} .
\end{align*}
Für die Pendellänge $L_2$ ergibt sich der Mittelwert
\begin{equation*}
    T_\text{s} = \SI{28.45 +- 0.07}{s}
\end{equation*}
für die Schwebungsdauer, daraus ergibt sich eine Frequenz
\begin{equation*}
    \omega_\text{s} = \SI{0.221 +- 0.005}{\frac{1}{s}} .
\end{equation*}
Die Schwebungsdauer lässt sich nach Gl.\eqref{eq:schwebungsdauer} auch mit Hilfe der in \ref{sec:kopplung} ermittelten Schwingungsdauern $T_+$ und $T_-$ berechnen zu
\begin{equation*}
    T_\text{s} = \SI{32.0 +- 5.0}{s} ,
\end{equation*}
die dazugehörige Frequenz ist
\begin{equation*}
    \omega_\text{s} = \SI{0.198 +- 0.029}{\frac{1}{s}} .
\end{equation*}
Aus den zuvor bestimmten Theoriewerten ergeben sich
\begin{align*}
    T_\text{s, th} = \SI{31 +-4}{s} \\
    \omega_\text{s, th} = \SI{0.204 +- 0.029}{\frac{1}{s}} 
\end{align*}
als Theoriewerte für die Schwebungsdauer und -frequenz.