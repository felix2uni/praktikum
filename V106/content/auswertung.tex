\section{Auswertung}
\label{sec:Auswertung}

Der erste Teil des Versuches wurde mit den Pendellängen
\begin{equation*}
    L_1 = \SI{50 +- 1}{cm} 
\end{equation*}
durchgeführt. Der Fehler wurde dabei mit 1 cm abgeschätzt. Da die gemessenen Werte für die Schwingungsdauer bei beiden Pendeln im Rahmen der Messungenauigkeit übereinstimmten,
wurden diese Pendellänge für die folgenden Teile beibehalten.
Der zweite Teil des Versuches wurde mit der Pendellänge
\begin{equation}
    L_2 = \SI{40 +- 1}{cm}
\end{equation}
durchgeführt.

\subsection{Justierung}

\begin{table}
	\centering
	\caption{Experimentell ermittelte Schwingungsdauern}
	\label{tab:justierung}
	\begin{tabular}{ S S | S S  }
		\toprule
		\multicolumn{2}{c}{$L_1$} & \multicolumn{2}{c}{$L_2$} \\
		\cmidrule(lr){1-2}\cmidrule(lr){3-4}
		{$ T_\text{a}\:/\: \si{s}$} & {$ T_\text{b}\:/\: \si{s}$} & {$ T_\text{a}\:/\: \si{s}$} & {$ T_\text{b}\:/\: \si{s}$}  \\
		\midrule
1.454 & 1.452 & 1.340 & 1.316 \\
1.448 & 1.448 & 1.340 & 1.366 \\
1.458 & 1.482 & 1.384 & 1.338 \\
1.460 & 1.436 & 1.368 & 1.350 \\
1.458 & 1.448 & 1.368 & 1.360 \\
1.442 & 1.460 & 1.368 & 1.346 \\
1.442 & 1.448 & 1.350 & 1.372 \\
1.448 & 1.452 & 1.350 & 1.356 \\
1.436 & 1.436 & 1.344 & 1.374 \\
1.452 & 1.454 & 1.344 & 1.350 \\

		\bottomrule
	\end{tabular}
\end{table}
Da bei der Durchführung die Dauer von 5 Schwingungen gemessen wurde, müssen die Messwerte durch 5 geteilt werden, um die Schwingungsdauer T zu erhalten.
Die so berechneten Werte für $T$ sind in Tabelle \ref{tab:justierung} dargestellt.
Durch die Mittelwertsformel 
\begin{equation}
    \label{eq:mittelwert}
    \bar{x} = \frac{1}{N} \sum_{n=1}^N x_n
\end{equation}
und die Formel für die Standardabweichung
\begin{equation}
    \label{eq:abweichung}
    \Delta x = \sqrt{\frac{1}{N-1} \sum_{n=1}^N (x_n-\bar{x})^2 }
\end{equation}
ergeben sich für $L_1$ die Mittelwerte
\begin{gather*}
    T_a = \SI{1.4498 +- 0.0025}{s} \\
    T_b = \SI{1.4520 +- 0.0040}{s}
\end{gather*}
%  TODO eqn
$T_a$ ist dabei die Periodendauer des ersten Pendels, $T_b$ die Periodendauer des zweiten.

\noindent Durch Gl.\eqref{eq:Periodendauer} kann für $L = \SI{50 +- 1}{cm} $ und $g = \SI{9.81}{\frac{m}{s^2}} $ der Theoriewert für die Schwingungsdauer zu
\begin{equation}
    T_\text{th} = \SI{1.419 +- 0.014}{s}
\end{equation}
bestimmt werden.
Der Fehler berechnet sich dabei nach
\begin{equation}
    \label{eq:theorie_fehler}
    \Delta T = \sqrt{ \left( \frac{1}{2} \sqrt{ \frac{1}{g l} } \right)^2 ( \Delta L)  ^2 } .
\end{equation}
Analog ergeben sich für $L_2$ die Werte:
\begin{gather}
    T_a = \SI{1.355 +- 0.004}{s} \\
    T_b = \SI{1.353 +- 0.005}{s}
\end{gather}
Der dazugehörige Theoriewert ist 
\begin{equation}
    T_\text{th} = \SI{1.269 +- 0.016}{s}
\end{equation}
Der Fehler wurde mit Gl.\eqref{eq:theorie_fehler} brerechnet.

\subsection{Bestimmung der Kopplungskonstante}
\label{sec:kopplung}
\begin{table}
	\centering
	\caption{Experimentell ermittelte Schwingungsdauern $T_+$ und $T_-$}
	\label{tab:TPlusMinus}
	\begin{tabular}{ S S | S S  }
		\toprule
		\multicolumn{2}{c}{$L_1$} & \multicolumn{2}{c}{$L_2$} \\
		\cmidrule(lr){1-2}\cmidrule(lr){3-4}
		{$ T_+\:/\: \si{s}$} & {$ T_-\:/\: \si{s}$} & {$ T_+\:/\: \si{s}$} & {$ T_-\:/\: \si{s}$}  \\
		\midrule
        1.466 & 1.374 & 1.338 & 1.280 \\
        1.448 & 1.406 & 1.356 & 1.292 \\
        1.482 & 1.418 & 1.386 & 1.286 \\
        1.482 & 1.402 & 1.350 & 1.320 \\
        1.470 & 1.408 & 1.340 & 1.316 \\
        1.460 & 1.436 & 1.360 & 1.276 \\
        1.454 & 1.414 & 1.350 & 1.340 \\
        1.470 & 1.378 & 1.358 & 1.294 \\
        1.458 & 1.418 & 1.368 & 1.322 \\
        1.470 & 1.386 & 1.366 & 1.288 \\

		\bottomrule
	\end{tabular}
\end{table}
\paragraph{1. Für $L_1$}
In Tabelle \ref{tab:TPlusMinus} sind die experimentell gemessenen Werte für $T_+$ und $T_-$ für beide Pendellängen aufgelistet.
Wie zuvor wurden dabei die gemessenen Zeiten durch $5$ geteilt, um die Periodendauer zu bestimmen. Durch Mittelwertsbildung und Bestimmung der
Standardabweichung ergeben sich:
\begin{gather}
    T_+ = \SI{1.4660 +- 0.0035}{s} \\
    T_- = \SI{1.404 +- 0.006}{s}
\end{gather}
Mit Gleichung \eqref{eq:kopplungskonstante} lässt sich daraus die Kopplungskonstante $K$ bestimmen zu 
\begin{equation}
    K = \num{0.043 +- 0.005} .
\end{equation}

\paragraph{2. Für $L_2$}
Die gemessenen Schwingungsdauern sind in Tabelle \ref{tab:TPlusMinus} zu sehen. Durch Mittelung der Werte ergeben sich folgende Schwingungsdauern:
\begin{gather}
    T_+ = \SI{1.357 +- 0.004}{s} \\
    T_- = \SI{1.301 +- 007}{s}
\end{gather}
Mit Gleichung \eqref{eq:kopplungskonstante} lässt sich die Kopplungskonstante $K$ bestimmen zu
\begin{equation}
    K = \num{0.042 +- 0.006} .
\end{equation}

\subsection{Messung der Schwebungsdauer}

\begin{table}
	\centering
	\caption{Experimentell ermittelte Schwebungsdauern $T_\text{s}$ und Schwingungsdauern $T$}
	\label{tab:schwebung}
	\begin{tabular}{ S S | S S  }
		\toprule
		\multicolumn{2}{c}{$L_1$} & \multicolumn{2}{c}{$L_2$} \\
		\cmidrule(lr){1-2}\cmidrule(lr){3-4}
		{$ T_\text{s}\:/\: \si{s}$} & {$ T \:/\: \si{s}$} & {$ T_\text{s}\:/\: \si{s}$} & {$ T \:/\: \si{s}$}  \\
		\midrule
		37.402 & 1.470 & 28.160 & 1.332  \\ 
		37.404 & 1.492 & 28.632 & 1.346  \\ 
		37.526 & 1.488 & 28.368 & 1.316  \\ 
		37.578 & 1.566 & 28.506 & 1.306  \\ 
		37.526 & 1.452 & 28.762 & 1.310  \\ 
		37.512 & 1.436 & 28.510 & 1.332  \\ 
		37.464 & 1.482 & 28.064 & 1.316  \\ 
		37.494 & 1.466 & 28.460 & 1.316  \\ 
		37.734 & 1.446 & 28.636 & 1.310  \\ 
		37.424 & 1.460 & 28.386 & 1.320  \\ 
		
		\bottomrule
	\end{tabular}
\end{table}

\paragraph{1. Für $L_1$: }
In Tabelle \ref{tab:schwebung} sind die gemessenen Schwebungsdauern für beide Pendellängen aufgelistet. Die Werte wurden wie zuvor durch 5 geteilt, um die Dauer einer Schwebung zu bestimmen.
Für die Pendellänge $L_1$ ergibt sich der Mittelwert
\begin{equation}
    T_s = \SI{37.506 +- 0.031}{s}
\end{equation}
für die Schwebungsdauer.
Die Schwebungsdauer lässt sich nach Gl.\eqref{eq:schwebungsdauer} auch mit Hilfe der in \ref{sec:kopplung} ermittelten Schwingungsdauern $T_+$ und $T_-$ berechnen zu
\begin{equation}
    T_s = \SI{33.0 +- 4.0}{s}
\end{equation}
Der Fehler wurde dabei gemäß der gausschen Fehlerfortpflanzung berechnet nach
\begin{equation}
    \Delta T_s = \sqrt{ \left( \frac{- T_-^2}{(T_+-T_-)^2} \right)^2 (\Delta T_+)^2 + \left( \frac{T_+^2}{ (T_+ - T_- )^2}  \right)^2 (\Delta T_-)^2 } .
\end{equation}

\paragraph{2. Für $L_2$: }
Für die Pendellänge $L_2$ ergibt sich der Mittelwert
\begin{equation}
    T_s = \SI{28.45 +- 0.07}{s}
\end{equation}
für die Schwebungsdauer.
Die Schwebungsdauer lässt sich nach Gl.\eqref{eq:schwebungsdauer} auch mit Hilfe der in \ref{sec:kopplung} ermittelten Schwingungsdauern $T_+$ und $T_-$ berechnen zu
\begin{equation}
    T_s = \SI{32.0 +- 5.0}{s} .
\end{equation}