\section{Durchführung}
\label{sec:Durchführung}
\subsection{Versuchsaufbau}
Für den Versuch werden zwei identische Stabpendel, bei denen aber die Masse des Stabes vernachlässigt werden soll, nebeneinander aufgebaut. Die Pendel haben im variablen Abstand $l$ von der Aufhängung eine zylinderförmige Masse $m$.
Zusätzlich haben die Stäbe der Pendel noch Löcher über die die beiden Pendel mit einer Feder verbunden werden können.

Um die Periodendauer zu messen wird eine Stoppuhr verwendet und die Messung der Pendellänge wird mit einem Maßband durchgeführt.
\subsection{Vorbereitung}
\label{sec:Vorbereitung}
\paragraph{1.} Zur Vorbereitung wird zunächst der Begriff der harmonische Schwingung diskutiert. Eine Schwingung wird als harmonisch bezeichnet, wenn die rücktreibende Kraft proportional zur 
Auslenkung und dieser entgegengesetzt ist. Die Bewegung lässt sich dann durch eine Sinus- oder Cosinusfunktion beschreiben \cite{demtroeder}.
\paragraph{2.} Es soll abgeschätzt werden, wie weit ein Pendel ausgelenkt werden darf, damit die Kleinwinkelnäherung \eqref{eq:kleinwinkel}
gilt. Die Kleinwinkelnäherung ist bis zu einem Winkel von etwa $\phi = \ang{10}$ gültig \cite{demtroeder}. Eine Pendel mit einer Länge von $l = \SI{70}{cm}$ kann daher um eine Strecke von
\begin{equation*}
    a = l \cdot \sin{\phi} \approx \SI{12.16}{cm}
\end{equation*}
ausgelenkt werden, bevor die Kleinwinkelnäherung nicht mehr gilt.

\subsection{Versuchsdurchführung}
\begin{itemize}
    \item Es wird für jedes einzelne Pendel zehn Mal die Periodendauer $T$ gemessen, indem die Zeit zwischen fünf Schwingungen gestoppt wird. Die Pendellänge der beiden Pendel soll dabei übereinstimmen. Um dies zu überprüfen, sollen die Periodendauern
        $T_1$ und $T_2$ mit einander verglichen werden. Weichen die gemessenen Periodendauern zwischen den Pendeln außerhalb der Messunsicherheit von einader ab müssen die Längen der Pendel angepasst werden.
    \item Im zweiten Teil des Versuches werden die beiden Pendel mittels einer Feder miteinader gekoppelt und die gegensinnige Schwingung $T_-$ sowie die gleichseitige Schwingung $T_+$. Es wird erneut zehn mal über Fünf Perioden gemessen.
    \item Danach muss noch die Schwebungsdauer $T_S$ und die Periodendauer $T$ für eine gekoppelte Schwingung gemessen werden, erneut jeweils 10 mal über fünf Perioden, bei der ein Pendel mit einer Auslenkung startet und eins aus der Ruhelage heraus.
    \item Nun müssen alle vorherigen Versuche erneut mit zwei veränderten Pendellängen durchgeführt werden.
\end{itemize}