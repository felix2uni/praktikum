\section{Diskussion}
\label{sec:Diskussion}

\subsection{Temperaturabhängigkeit der Viskosität}

Trotz der vielen Messungenauigkeiten, vor allem bei der Fallzeit $t$, aber
beim Gewicht $m$ und dem Radius $r$ der Kugeln, sind die Messfehler der
Koeffizienten in der Andraedschen Gleichung relativ klein.
Die prozentualen Abweichungen $\Delta x$ der Viskositäts-Werte
$\eta_\text{mess}(T)$ von den
Literaturwerten $\eta_\text{lit}(T)$ \cite{TfCuP}
bleiben über das gesamte Temperaturspektrum relativ konstant bei um $\SI{24}{\percent}$,
wie an Tabelle \ref{tab:compare} erkennbar ist.
Der Fehler ist vor allem dadurch zu erklären, dass das manuelle Stoppen mit der Stoppuhr sehr ungenau ist.


\begin{table}
  \centering
  \input{build/compare.tex}
  \caption{Messwerte der Fallzeit der kleinen Kugel bei Raumtemperatur.}
  \label{tab:compare}
\end{table}


\subsection{Reynolds-Zahl}

Die berechnete Reynolds-Zahl liegt mit

\begin{equation}
  Re = \num{6.75e-6}
\end{equation}
eindeutig unter dem kritischen Wert

\begin{equation}
  Re_\text{krit} = 2300.
\end{equation}
Damit ist die Strömung laminar.