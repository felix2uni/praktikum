\section{Theorie}
Ein Körper, welcher sich durch eine Flüssigkeit bewegt, wird durch eine Reibungskraft abgebremst. Für eine Kugel, die sich so durch eine Flüssigkeit bewegt, dass keine Wirbel 
entstehen lässt sich die Stokessche Reibung berechnen nach
\begin{equation}
    F_\text{R} = 6 \pi \eta v r .
\end{equation}
$v$ ist dabei die Geschwindigkeit der Kugel, $r$ dessen Radius und $\eta$ ist die dynamische Viskositöt der Flüssigkeit. Die dynamische Viskosität ist eine Materialeigenschaft, 
welche durch ein Kugelfallviskosimeter ermittelt werden kann. Beim Kugelfallviskosimeter fällt eine Kugel durch ein mit einer Flüssigkeit gefülltes Rohr, welches nur einen 
geringfügig größeren Durchmesser als die Kugel hat. Beim Fall der Kugel stellt sich dann ein Gleichgewicht aus der Schwerkraft, der Auftribeskraft der Kugel und der Reibungskraft 
ein. Die Viskosität der Flüssigkeit lässt sich dann berechnen zu
\begin{equation}
    \label{eq:viskositaet}
    \eta = K (\rho_\text{K} - \rho_\text{Fl}) \cdot t .
\end{equation}
$\rho_\text{K}$ ist dabei die Dichte er Kugel, $\rho_\text{Fl}$ die Dichte der Flüssigkeit und $t$ ist die Fallzeit. $K$ ist eine Apparaturkonstante. Diese beschreibt unter anderem 
die Fallhöhe und die Kugelgeometrie \cite{V207}.

\noindent Im Allgemeinen ist die Viskosität einer Flüssigkeit abhängig von der Temperatur. Die Temperaturabhängigkeit kann durch die Andradesche Gleichung beschrieben werden:
\begin{equation}
    \label{eq:andradescheGl}
    \eta (T) = A \exp{\frac{B}{T}}
\end{equation}
$A$ und $B$ sind dabei Materialkonstanten \cite{V207}.

\noindent In diesem Versuch werden außerdem einige Strömungseigenschaften betrachtet. Strömungen lassen sich allgemein in laminare und turbulente Strömungen einteilen. Bei einer 
laminaren Strömung kreuzen sich die einzelnen Stromlinien nicht, bei turbulenten Strömungen treten Verwirbelungen auf. Ein Maß dafür, ob es sich um eine laminare oder eine 
turbulente Strömung handelt, ist die Reynoldszahl. Sie wird berechnet nach
\begin{equation}
    \label{eq:reynold}
    \text{Re} = \frac{v l \rho}{\eta}.
\end{equation}
$v$ ist dabei die Strömungsgeschwindigkeit, $\rho$ und $\eta$ die Dichte und dynamische Viskosität des Strömungsfluids und $l$ ist die charakteristische Länge. In diesem Versuch
 wird diese mit dem Durchmesser der Kugel abgeschätzt \cite{viskositaet}.