\section{Auswertung}
\label{sec:Auswertung}
\newcommand{\kleineKugelDurchmesser}{\SI{15.61}{\milli\meter} }
\newcommand{\kleineKugelRadius}{\SI{0.007805}{\milli\meter} }
\newcommand{\kleineKugelMasse}{\SI{0.0044531}{\kilo\gram} }
\newcommand{\kleineKugelDichte}{\SI{2235.91}{\kilo\gram\per\cubic\meter} }

\newcommand{\grosseKugelRadius}{\SI{0.00789}{\milli\meter} }
\newcommand{\grosseKugelDurchmesser}{\SI{0.01578}{\milli\meter} }
\newcommand{\grosseKugelMasse}{\SI{0.0049528}{\kilo\gram} }
\newcommand{\grosseKugelDichte}{\SI{2407.31}{\kilo\gram\per\cubic\meter} }

Bei der kleinen Kugel wurde ein Durchmesser von \kleineKugelDurchmesser beziehungsweise ein Radius von \kleineKugelRadius gemessen.
Daraus lässt sich zusammen mit einer Masse von \kleineKugelMasse eine Dichte von
\begin{equation}
  \rho_\text{k}
  = \frac{m_\text{k}}{V_\text{k}}
  = \frac{m_\text{k}}{\frac{4}{3}\pi r_\text{k}^3}
  = \frac{\kleineKugelMasse}{\frac{4}{3}\pi\times(\kleineKugelRadius)^3}
  = \kleineKugelDichte
\end{equation}
berechnen.
Äquivalent lässt sich für die große Kugel mit einem Radius von \grosseKugelRadius und einer Masse von \grosseKugelMasse die Dichte berechnen.
\begin{equation}
  \rho_\text{k}
  = \frac{m_\text{g}}{V_\text{g}}
  = \frac{m_\text{g}}{\frac{4}{3}\pi r_\text{g}^3}
  = \frac{\grosseKugelMasse}{\frac{4}{3}\pi\times(\grosseKugelRadius)^3}
  = \grosseKugelDichte.
\end{equation}

\begin{table}
  \centering
  \input{build/kleine_kugel.tex}
  \caption{Messwerte der Fallzeit der kleinen Kugel bei Raumtemperatur.}
  \label{tab:kleine_kugel}
\end{table}

\begin{table}
  \centering
  \input{build/grosse_kugel.tex}
  \caption{Messwerte der Fallzeit der großen Kugel bei Raumtemperatur.}
  \label{tab:grosse_kugel}
\end{table}


\noindent In Tabelle \autoref{tab:kleine_kugel} sind die Messwerte der Fallzeit der kleinenKugel aufgeführt.
Diese werden gemittelt, sodass mit einer Zeit
\begin{equation}
  t_{k} = \SI{12.80}{\second}
\end{equation}
weiter gerechnet werden kann.


\noindent Mit der gegebenen Apparaturkonstante

\begin{equation}
  K_\text{k} = \SI{0.07640e-6}{\pascal\cubic\meter\per\kilo\gram}
\end{equation}
für die kleine Kugel und der Dichte von Wasser \cite{TfCuP}

\begin{equation}
  \rho_\text{W} = \SI{998.2}{\kilo\gram\per\cubic\meter}
\end{equation}
folgt bei Raumtemperatur $T=\SI{292,15}{\kelvin}$ über die Formel
\begin{equation}
  \eta_\text{W} = (\rho_\text{k}-\rho_\text{W})K_\text{k} \cdot t_\text{k}
  \label{eqn:Viskositaet}
\end{equation}
die Viskosität des destillierten Wassers

\begin{equation}
  \eta_\text{W} = \SI{1.21(1)e-3}{\pascal\second}.
  \label{eqn:Viskosi}
\end{equation}


\subsection{Apparaturkonstante}

Die Messung der Temperaturabhängigkeit wird mit der großen Kugel durchgeführt,
weshalb zunächst die Apparaturkonstante für diese Kugel bestimmt werden muss.
Die Werte für die Masse und den Radius der großen Kugel betragen

\begin{align}
  m_\text{g} & = \grosseKugelMasse & r_\text{g} & =\grosseKugelRadius.
\end{align}
Das führt zu der Dichte

\begin{equation}
  \rho_\text{g} = \frac{m_\text{g}}{V_\text{g}} = \frac{m_\text{g}}
  {\frac{4}{3}\pi r_\text{g}^3} = \SI{2407.31}{\kilo\gram\per\cubic\meter}.
  \label{eqn:rhog}
\end{equation}
In Tabelle \ref{tab:grosse_kugel} sind die Messwerte der Fallzeit der großen
Kugel aufgeführt. Der Mittelwert der zehn Werte beträgt

\begin{equation}
  t_\text{g} = \SI{48.45}{\second}.
\end{equation}
Über die berechneten Dichten, die Fallzeit und die Viskosität $\eta_\text{W}$
von Wasser folgt bei Raumtemperatur die Apparaturkonstante

\begin{equation}
  K_\text{g} = \frac{\eta_\text{W}}{t(\rho_\text{g}-\rho_\text{W})}
  = \SI{1.77e-8}{\pascal\cubic\meter\per\kilo\gram}
\end{equation}
für die große Kugel.


\subsection{Temperaturabhängigkeit der Viskosität}

\begin{table}
  \centering
  \input{build/grosse_kugel_temperatur.tex}
  \caption{Messwerte der Fallzeit der großen Kugel bei verschiedenen Temperaturen.}
  \label{tab:grosse_kugel_temprature}
\end{table}

\noindent In \autoref{tab:grosse_kugel_temprature} sind die Messwerte der Fallzeiten für verschiedene Temperaturen angegeben.


\noindent Aus den Werten für die Fallzeit $t$ kann über die Formel \eqref{eq:viskositaet}
die Viskosität für die jeweiligen Temperaturen bestimmt werden.
Die Messwertpaare werden dann mittels einer Exponentialfunktion gefittet.
In Abbildung \ref{fig:grosse_kugel_temperatur} sind die Messwerte und die Ausgleichsfunktion
jeweils mit ln$(\eta)$ und $\frac{1}{T}$ dargestellt.

\begin{figure}[H]
  \centering
  \includegraphics[]{build/grosse_kugel_temperature.pdf}
  \caption{Viskosität $\eta$ in Abhängigkeit von $t$}
  \label{fig:grosse_kugel_temperatur}
\end{figure}

\noindent Aus diesen Werten lässt sich nun eine Ausgleichsfunktion fitten und daraus die Koeffizienten der Andradeschen Gleichung ablesen.


\begin{equation}
  A = \num{5.12e-4}
\end{equation}
und

\begin{equation}
  B = (\num{184}).
\end{equation}
Damit lautet die Funktion der Viskosität von der Temperatur
\begin{equation}
  \eta(T) = (\num{5.12e-4}) \symup{e}^{\frac{\num{184}}{T}\si{\second}}
  \si{\pascal\second}.
\end{equation}


\subsection{Reynolds-Zahl}

Die Reynolds-Zahl des destillierten Wassers kann mit Hilfe der Viskosität aus
Gleichung \eqref{eq:viskositaet}, der Dichte des Wassers \cite{TfCuP},
der Fallrohrdicke, die etwa dem Durchmesser der großen Kugel

\begin{equation}
  d_\text{g} = 2 r_\text{g} = \grosseKugelDurchmesser
\end{equation}
entspricht, und der Geschwindigkeit der großen Kugel

\begin{equation}
  v_\text{g} = \frac{0.05}{48,45}\si{\meter\per\second} =
  \SI{1.03e-3}{\meter\per\second}
\end{equation}
bestimmt werden.
Mit Gleichung \eqref{eq:reynold} folgt

\begin{equation}
  Re = \num{6.75e-6}.
\end{equation}