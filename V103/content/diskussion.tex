\section{Diskussion}
\label{sec:Diskussion}
Das Material des zu untersuchenden Stab hat große optische Ähnlichkeiten zum Material Kupfer. Die aus den Messwerten berechnete Dichte des runden Stabes liegt bei $\rho=8.89\frac{g}{cm^3}$ und ist damit mit einer Abweichung von $0.33\%$ relativ nahe am Literaturwert von Kupfer mit $\rho=8.92\frac{g}{cm^3}$.
\noindent Die berechnete Dichte des Stabes mit quadratischer Grundfläche liegt mit $\rho=8.93\frac{g}{cm^3}$ und einer Abweichung von $0.11\%$ sehr nahe am Literaturwert. Somit kann davon ausgegangen werden, dass es sich bei beiden Stäben um Kupfer handelt.
\noindent Der theoretische Wert des Elastizitätsmodul bei Kupfer beträgt $\qty{120}{\giga\pascal}=\qty{1.2e11}{\newton\per\meter\squared}$
Die aus den Messwerten der einseitigen Einspannung ermittelten Elastizitätsmodule sind mit Abweichungen von 7.6\% und 4.7\% sehr nahe am Literaturwert für Kupfer mit  für Kupfer mit $E=1.2\times10^{11}$.
\noindent Im Gegensatz dazu sind die Ergebnisse bei den doppelten einspannungen deutlich schlechter.
\noindent Das aus den Messwerten der linken Seite des doppelt eingespannten rechteckigen Stabes berechnete Elastizitätsmodul $E=(2.75 \pm 0.12) \times 10^{11} \unit{\newton\per\meter\squared}$ hat eine Abweichung von 56.3\% vom Literaturwert.
\noindent und das aus den Messwerten der rechten Seite des doppelt eingespannten rechteckigen Stabes berechnete Elastizitätsmodul $E=(4.00 \pm 0.40) \times 10^{11} \unit{\newton\per\meter\squared}$ hat eine Abweichung von 69.8\% vom Literaturwert.
\noindent Die Ergebnisse des runden Stabes sind mit $E=(3.23 \pm 0.10) \times 10^{11} \unit{\newton\per\meter\squared}$ und einer Abweichung von 62.8\% für die linke Seite und $E=(5.6 \pm 0.6) \times 10^{11} \unit{\newton\per\meter\squared}$ mit einer Abweichung von 78.5\% für die rechte Seite liegen auch sehr weit vom Literaturwert entfernt.
\noindent Ein Grund dafür könnte die deutlich kleinere Biegung bei der doppelten Einspannung sein wodurch die Messunsicherheiten der Messuhren deutlich stärker ins Gewicht fallen.
\noindent Ebenso treten bei beim verrechnen von werten aus unterschiedlichen Größenordnungen deutlich größere Fehler auf.