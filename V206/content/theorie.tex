\section{Theorie}
\label{sec:Theorie}

Um einen Wärmefluss von einem kälteren Behältnis in ein wärmeres Behältnis zu realisieren muss Arbeit aufgewandt werden.
Ein Weg diese Arbeit aufzuwenden ist eine Wärmepumpe, diese wird später noch genauer betrachtet.
Um die Berechnungen zu vereinfachen wird die Wärmepumpe im folgenden erst einmal als idealisiert betrachtet.
Die Güteziffer $\nu$ beschreibt das Verhältnis zwischen der entnommenen Wärmemenge $Q_2$ und der dazu aufgewendeten Arbeit $A$, die Summe dieser beiden Größen ist die gesamte im warmen Behältnis aufgenommene Wärmemenge.


\begin{equation}
  Q_1 = Q_2 + A
  \label{eqn:q1}
\end{equation}
\begin{equation}
  \nu = \frac{Q1}{A}
  \label{eqn:nu}
\end{equation}

\noindent Nach dem 2.HS der Thermodynamik lässt sich zudem die Beziehung zwischen den Wärmemengen und Temperaturen der beiden Reservoire durch folgende Formel ausdrücken:
\begin{equation}
  \frac{Q_1}{T_1} - \frac{Q_2}{T_2} = 0
  \label{eqn:2HS}
\end{equation}
Für die Gültigkeit dieser Formel muss jedoch gelten, dass der stattfindende Übertragungsprozess reversibel sei. Somit müsste die aufgewandte mechanische Energie jederzeit vollständig zurückgewonnen werden können.
Da es sich dabei um eine idealisierte Annahme handelt, die in der Realität nie zutrifft, muss \autoref{eqn:2HS} umformuliert werden:
\begin{equation}
  \frac{Q_1}{T_1} - \frac{Q_2}{T_2} > 0
  \label{eqn:2HS1}
\end{equation}
Aus den Gleichungen \ref{eqn:q1} bis \ref{eqn:2HS1} folgt somit:
\begin{align}
  \nu_{id}   & = \frac{Q_1}{A} = \frac{T_1}{T_1 - T_2} \label{eqn:A1.2} \\
  \nu_{real} & < \frac{Q_1}{A} = \frac{T_1}{T_1 - T_2} \label{eqn:A1.3}
\end{align}
Die Gleichungen \ref{eqn:A1.2} und \ref{eqn:A1.3} zeigen, dass eine Wärmepumpe umso effektiver eingestuft
werden kann, je kleiner die Differenz zwischen T1 und T2
ist.
\subsection*{Bestimmung der realen Güteziffer $\nu$}
Mit dem Werten von $T_1$ kann nun die pro Zeiteinheit gewonnene Wärmemenge berechnet werden:
\begin{align}
  \frac{\increment Q_1}{\increment t} & = (m_1c_W + m_kc_k) \frac{\increment T_1}{\increment t} \\
  \nu                                 & = \frac{\increment Q_1}{\increment t N}
  \label{eqn:Gueteziffer}
\end{align}
$m_1c_w$ und $m_k c_k$ entsprechen dabei den Wärmekapazitäten der Kupferschlange und des Eimers.
Für die Güteziffer wird noch N als die zeitlich gemittelte Leistung benötigt.
subsection{Bestimmung des Massendurchsatzes}
Mit den Werten von $T_2$ und der Verdampfungswärme $L$ kann nun der Massendurchsatz $\increment m $
berechnet werden:

\begin{align}
  \label{eqn:Massendurchsatz}
  \frac{\increment Q_2}{\increment t} & = (m_2 c_W + m_k c_k) \frac{\increment T_2}{\increment t} \\
  \frac{\increment Q_2}{\increment t} & = L \frac{\increment m}{\increment t}
\end{align}

\subsection{Bestimmung der mechanischen Kompressorleistung \texorpdfstring{$N_{mech}$}{t} }

Um die mechnanische Kompresorleistung $N_{mech}$ zu bestimmen muss vorher die vom Kompressor aufgebrachte
Arbeit zur Komprimierung des Volumens $V_a$ auf das Volumen $V_b$ berechnet werden:
\begin{equation}
  \label{eqn:Am}
  A_m = \frac{1}{\kappa - 1} \left( p_b \sqrt[\kappa]{\frac{p_a}{p_b}} - p_a \right) V_a
\end{equation}
Für den Kompressor wird nun näherungsweise angenommen, dass es sich um eine adiabatische Komprimierung handelt,
sodass die Poisson-Gleichung als Zusammenhang zwischen Druck und Volumen gilt:
\begin{equation}
  p_a V^{\kappa}_a = p_b V^{\kappa}_b = p V^{\kappa} .
\end{equation}
Mit der Dichte $\rho$ im gasförmigen Zustand, also beim Druck $p_a$, kann nun $N_{mech}$ berechnet werden:
\begin{equation}
  \label{eqn:Nmech}
  N_{mech} = \frac{\increment A_m}{\increment t} =  \frac{1}{\kappa - 1} \left( p_b \sqrt[\kappa]{\frac{p_a}{p_b}} - p_a \right) \frac{1}{\rho} \frac{\increment m}{\increment t}
\end{equation}

